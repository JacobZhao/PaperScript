\documentclass[journal]{IEEEtran}

%%%% Using Latex Packages %%%%


\usepackage{cite}
\usepackage{amsmath,amssymb,amsfonts}
\usepackage{algorithmic}
\usepackage{subcaption}
\usepackage{color}
\usepackage{graphicx}
\usepackage{xcolor}
\usepackage{enumerate}
\usepackage{multirow}
\usepackage{amsmath}
\usepackage{amssymb}
\usepackage[utf8]{inputenc}
\usepackage[ruled]{algorithm2e}
\usepackage[colorinlistoftodos]{todonotes}
\def\BibTeX{{\rm B\kern-.05em{\sc i\kern-.025em b}\kern-.08em
    T\kern-.1667em\lower.7ex\hbox{E}\kern-.125emX}}
\newtheorem{definition}{Definition}
\renewcommand{\topfraction}{1.0}
\renewcommand{\bottomfraction}{1.0}
\renewcommand{\dbltopfraction}{1.0}
\renewcommand{\textfraction}{0.01}
\renewcommand{\floatpagefraction}{1.0}
\renewcommand{\dblfloatpagefraction}{1.0}
\setcounter{topnumber}{5}
\setcounter{bottomnumber}{5}
\setcounter{totalnumber}{10}


\hyphenation{op-tical net-works semi-conduc-tor}


\begin{document}

\title{Efficient Feasibility Checking Algorithm of Photovoltaic Array Reconfiguration}

\author{Dafang Zhao,
        Fukohito Ooshita,~\IEEEmembership{Member,~IEEE}
        and~Michiko~Inoue,~\IEEEmembership{Member,~IEEE}}% <-this % stops a space

% \thanks{M. Shell was with the Department
% of Electrical and Computer Engineering, Georgia Institute of Technology, Atlanta,
% GA, 30332 USA e-mail: (see http://www.michaelshell.org/contact.html).}% <-this % stops a space
% \thanks{J. Doe and J. Doe are with Anonymous University.}% <-this % stops a space
% \thanks{Manuscript received April 19, 2005; revised August 26, 2015.}}

% note the % following the last \IEEEmembership and also \thanks - 
% these prevent an unwanted space from occurring between the last author name
% and the end of the author line. i.e., if you had this:
% 
% \author{....lastname \thanks{...} \thanks{...} }
%                     ^------------^------------^----Do not want these spaces!


% make the title area
\maketitle

% As a general rule, do not put math, special symbols or citations
% in the abstract or keywords.
\begin{abstract}
  Power generation efficiency of photovoltaic (PV) systems is significantly affected buy partial shading and PV cell damage.
  Partial shading or PV cell damage induces mismatched power generation among PV panels.
  Conducted bypass diodes under mismatch conditions result in loss of efficiency in power generation.
  Mismatched PV array can be recovered by re-configuring electrical connections among PV panels in it.
  In this paper, a feasibility check problem of PV panel reconfiguration is introduced.
  This problem identifies whether a connection among PV panels can be configured from a given PV module level solution.
  Proposed algorithm evaluated by comparison with the exhaustive search through random shading distributed PV array.
  The experimental results demonstrate that proposed algorithm can identify feasible configurations more than 49,000 times faster than the exhaustive search with around 0.5\% errors.
\end{abstract}

% Note that keywords are not normally used for peerreview papers.
\begin{IEEEkeywords}
PV reconfiguration, partial-shading, mismatch, feasibility, heuristic
\end{IEEEkeywords}

\IEEEpeerreviewmaketitle



\section{Introduction}
\IEEEPARstart{I}{n} recent years, the use of green and renewable energy sources has been increased with the aim to reduce fossil fuel depletion and environment pollution.
Photovoltaic (PV) energy is one of the most promising emerging technologies.
PV market growth by improvements of converting unlimited solar energy into electrical energy as well as the cost reductions of PV panels.

The use of PV systems for power generation brings many challenges.
Due to the nature of PV cell, which is the basic component of PV array.
PV system easily suffers from various forms of system faults, which include physical damage, temperature in-homogeneity, or partial shading.
Unlike cell damage or other system faults, partial shading sources from cloud, dust or snow are very hard to prevent and predict.
Thus, when PV cell could not uniformly generate power when they experience different irradiance or been damaged.
This unbalanced working scenario will lead whole system mismatch.
Mismatch condition might accelerate heating or aging of PV cells and furthermore hinder operation of maximum power point tracking (MPPT) algorithm, especially when the PV array output P-V curve becomes non-convex\cite{islam2018performance}.

For several series connected PV cells, shaded or damaged cell causing normal cells to produce higher voltages that may reverse bias of ``bad'' cells.
When a large number of series connected cells cause a huge reverse bias across shaded cells, leading to large dissipation of energy in the ``bad'' cells.
This huge energy dissipation occurring in a small area might get overheating or burning of PV cells, or ``hot-spots''.
To protect PV cells from ``hot-spots'', bypass diode is used to circumvent concentrated energy dissipation.
However, the operation of bypass diode will cause several stop delivering power and generate multiple local maximum power points\cite{Orozco-Gutierrez2016}.
Furthermore, strand bypass diode can not complete eliminate hot-spotting\cite{kim2015reexamination}.

In order to improve PV system power generation efficiency and protect PV cells from damage, an efficient and effectively PV system manage method is worth to investigate.

\section{Conclusion}
The conclusion goes here.



\appendices
\section{Proof of the First Zonklar Equation}
Appendix one text goes here.

% you can choose not to have a title for an appendix
% if you want by leaving the argument blank
\section{}
Appendix two text goes here.


% use section* for acknowledgment
\section*{Acknowledgment}


The authors would like to thank...


% Can use something like this to put references on a page
% by themselves when using endfloat and the captionsoff option.
\section*{Temp}
It is well known that mismatch due to partial shading, soiling, or ageing causes significant losses
in the energy yield of photovoltaic (PV) systems [1,2]. Furthermore, mismatches may hinder operation
of maximum power point (MPP) tracking algorithms, especially if the power versus output voltage
characteristic becomes nonconvex [3]. It has also been shown that, even with commonly used bypass
diodes, mismatched cells may become reverse-biased and dissipate power, producing an undesired
cell temperature rise or hot spot [4,5]. This may lead to accelerated ageing and reduced reliability of
the PV system


\bibliographystyle{ieeetr}
\bibliography{library}


\end{document}


