\documentclass[conference]{IEEEtran}
\IEEEoverridecommandlockouts
% The preceding line is only needed to identify funding in the first footnote. If that is unneeded, please comment it out.
\usepackage{cite}
\usepackage{amsmath}
\usepackage{amssymb}
\usepackage{amsfonts}
\usepackage{algorithmic}
\usepackage{graphicx}
\usepackage{textcomp}
\usepackage{xcolor}
\usepackage{graphicx}
\usepackage{subfigure}
\def\BibTeX{{\rm B\kern-.05em{\sc i\kern-.025em b}\kern-.08em
    T\kern-.1667em\lower.7ex\hbox{E}\kern-.125emX}}
\begin{document}

\title{Photovoltaic System Reconfiguration strategy for mismatch condition\\
{\footnotesize \textsuperscript{}}
%\thanks{Identify applicable funding agency here. If none, delete this.}
}

\author{\IEEEauthorblockN{1\textsuperscript{st} Dafang Zhao}
\IEEEauthorblockA{\textit{dept. name of organization (of Aff.)} \\
\textit{name of organization (of Aff.)}}
\and
\IEEEauthorblockN{2\textsuperscript{nd} Given Name Surname}
\IEEEauthorblockA{\textit{dept. name of organization (of Aff.)} \\
\textit{name of organization (of Aff.)}}
\and
\IEEEauthorblockN{3\textsuperscript{rd} Given Name Surname}
\IEEEauthorblockA{\textit{dept. name of organization (of Aff.)} \\
\textit{name of organization (of Aff.)}}}

\maketitle

\begin{abstract}
This document is a model and instructions for \LaTeX.
This and the IEEEtran.cls file define the components of your paper [title, text, heads, etc.]. *CRITICAL: Do Not Use Symbols, Special Characters, Footnotes, 
or Math in Paper Title or Abstract.
\end{abstract}

\begin{IEEEkeywords}
component, formatting, style, styling, insert
\end{IEEEkeywords}

\section{Introduction}

As the world of fossil energy constantly exhausted and the increasingly serious environmental pollution, the research and utilization of renewable energy and green energy have become maintain necessary means of survival and development of the human. Photovoltaic (PV) energy received significant attention since it has unlimited energy and easy to be scaled up. Thanks to extensive technology and research on photovoltaic energy generation, large scale photovoltaic energy generation system have been deployed into many practical application. But due to PV arrays are senstive to shading and PV cell's fault or aging. That means when interconnection of PV cells or modules do no have identical properties or experience different conditions from one another. PV arrays are in mismatch condition. In order to avoid mismatch condition damage PV cells, we proposed an algorithm that can re-configurate photovoltaic arrays to minimize mismatch loss.

In this paper, we using non-uniform irradiance levels to represent mismatch condition and analyzes the efficiency of a PV system under different shaded working condition is presented. When photovoltaic arrays operating in non-uniform irradiance levels may present multiple local maximum power points (MPPs) \cite{b1}, which been generated by turning on bypass diodes. By changing electrical connection among the panels to prevent activate bypass diodes is a recent appealing solution\cite{b2}.

The main difficulty facing the reconfiguration problem is that some or even all panels can be subjected to partial shading, so there may be more than one MPP for each panel. Reconfiguration strategy need also consider to group PV modules which provided high power separately\cite{b3} \cite{b4}.

This procedure enables to detect panel's operating conditions, in more than two-strings, receiving different irradiance levels. The reconfiguration algorithm will analyzes panels' working conditions and reorganizes panels into different strings by different irradiance levels. However, in sparse of mismatching conditions, distribution of panels among different irradiance levels are not significant. For that, by increasing number of strings in the PV array and using exhaustive search can be a solution\cite{b3}.Another approach to optimize photovoltaic arrays is using genetic algorithm\cite{b5}. However, computing cost is too significant, and this algorithm can't detect best configuration precisely.



\section{Assumption of a PV array} \label{assum}
In this paper, we using following definitions of PV arrays, modules, strings and panels. A PV array formed by several parallel connected PV strings, and string is several series connected PV panels. For a PV panel, formed by two or three PV modules connected in series with bypass diodes. An equivalent connection as showed in Fig. \ref{fig2}.
\begin{figure}[htbp]
\centerline{\includegraphics{fig2.png}}
\caption{Definition of PV array and internal components.}
\label{fig2}
\end{figure}

The algorithm we proposed based on following assumptions.

\begin{itemize}
\item The current versus voltage (\textit{I-V}) curve of each panel calculated by algorithm presented in \cite{b6}. This algorithm will analyzes panel's \textit{I-V} curve sample and coordinate to maximum or minimum power point in \textit{P-V} curve.
\item All the panels in PV array have same number (\textit{N}) of modules. For particular module, using (\textit{$V_{mpp_n}$, $I_{mpp_n}$}) to identify MPP voltage and current by index \textit{n}. Those parameter can be directly estimated by the prosess provided in\cite{b7}.
\end{itemize}

Furthermore, it is also assumed that for each string it has same number of panels in PV array. Means every string in PV array have same length. String's length are identical based on when they connected in parallel, a string has more panels may cause current back flow into other strings which have less panels\cite{b8}.
\begin{figure}[htbp]
\centerline{\includegraphics{fig2.png}}
\caption{Flowchart of reconfiguration algorithm.}
\label{fig2}
\end{figure}


\section{Reconfiguration Algorithm}
The general steps of reconfiguration algorithm are presented in Fig. \ref{fig2}. The first step is determine each PV module's \textit{$V_{mpp}$} and \textit{$I_{mpp}$} by using algorithm provided in \cite{b7} \cite{b9}, and calculate MPP current and voltage candidates of PV array. When there are more than two panels connected in series, for a string MPPs is not straightforward. The MPP current and voltage candidates can be evaluated though a procedure presented in \cite{b10}. For string MPP current candidates, \textit{$I_{mpp_n}$} values' different less than 5\% are assumed to be equal. For string MPP voltage candidates, \textit{$V_{mpp_n}$}  can be calculated though multiplay number of active modules (\textit{$N_a$}) by average MPP voltages (\textit{$\bar V_{mpp}$}) with $\pm$18\% error \cite{b10}. Afterward, determine real number of working modules (\textit{$Q_{M_n}$})  per string by applying method in \cite{b10}. Next, find MPP candidates by multiplying current candidates and voltage candidates which determine by \textit{$Q_{M_n}$}. Due to \textit{$V_{mpp}s$} and \textit{$I_{mpp}s$} can indicate shadowing distribution among PV array.  Then grouping panels into different shadow distribution.  

After this procedures is conducted for PV array, all panels will be organized into many groups that from un-shadowed or uniform shadow to fully shadowed group. However, if just simply grouping panels by shadow distribution conditions it may cause electrical connection overhead or unable calculate optimal configuration. To further reconfigure PV system into a better configuration, the replacement part of algorithm will proceed as follow:
\begin{itemize}
\item Sorting group by different shadow distribution conditions, select panels from first group into a PV string which working on high current level.
\item If selected panels' working modules ( \textit{$Q_{M_n}^{*}$} ) less than \textit{$Q_{M_n}$} , select panels from next irradiance level group.
\item If selected panels' \textit{$Q_{M_n}^{*}$} more than \textit{$Q_{M_n}$}, re-select panels to adjust \textit{$Q_{M_n}^{*}$} equal to \textit{$Q_{M_n}$} if it is possible. 
\end{itemize}

\begin{figure}[htbp]
\centerline{\includegraphics{fig2.png}}
\caption{Flowchart of reconfiguration algorithm.}
\label{fig3}
\end{figure}

This algorithm will better understood by applying to shadow condition showed in Fig. \ref{fig3}. This PV system is composed with 9 PV panels connected into 3 strings as showed in Fig.\ref{fig3}. Irradiance levels, \textit{$I_{mpp}$} and \textit{$V_{mpp}$} of each module are given in Table \ref{tab} 

\begin{table}[htbp]
\caption{Table Type Styles}
\begin{center}
\begin{tabular}{|c|c|c|c|}
\hline
\textbf{Table}&\multicolumn{3}{|c|}{\textbf{Table Column Head}} \\
\cline{2-4} 
\textbf{Head} & \textbf{\textit{Table column subhead}}& \textbf{\textit{Subhead}}& \textbf{\textit{Subhead}} \\
\hline
copy& More table copy$^{\mathrm{a}}$& &  \\
\hline
\multicolumn{4}{l}{$^{\mathrm{a}}$Sample of a Table footnote.}
\end{tabular}
\label{tab}
\end{center}
\end{table}
The example refer to a three-strings PV array, each string contain with 3 panels, each one made of three identical PV modules (\textit{N}=3). In this example, it is assumed that input range of DC-DC converter and MPPT device defined by \textit{$V_{min}$} = 200V and \textit{$V_{max}$} = 500V. MPP current candidates and MPP voltage candidates are present in ( \ref{e1} ) and ( \ref{e2} ), respectively. Additional, number of real working module at MPP current candidates are given in ( \ref{e3} ).
\begin{equation}
\begin{aligned}
\textit{$I_{mpp_n}$} = [&\{  \text{$0.5_{(9)}$}, \text{$0.5_{(9)}$}, \text{$0.5_{(9)}$}   \}, \{ \text{$0.5_{(9)}$}, \text{$0.5_{(9)}$}, \text{$2_{(11)}$} \},  \\ &\{\text{$0.5_{(9)}$}, \text{$0.5_{(9)}$}, \text{$3_{(4)}$} \}, \{\text{$0.5_{(9)}$}, \text{$2_{(11)}$}, \text{$2_{(11)}$} \}, \\ & \{\text{$0.5_{(9)}$}, \text{$2_{(11)}$}, \text{$3_{(4)}$} \}, \{\text{$0.5_{(9)}$}, \text{$3_{(4)}$}, \text{$3_{(4)}$} \}, \\ &\{\text{$2_{(11)}$}, \text{$2_{(11)}$}, \text{$2_{(11)}$} \}, \{\text{$2_{(11)}$}, \text{$2_{(11)}$}, \text{$3_{(4)}$} \}, \\ & \{\text{$2_{(11)}$}, \text{$3_{(4)}$}, \text{$3_{(4)}$} \}, \{\text{$3_{(4)}$}, \text{$3_{(4)}$}, \text{$3_{(4)}$} \}] \pm 5\%  \label{e1}
\end{aligned}
\end{equation}
\begin{equation}
\begin{aligned}
\textit{$V_{mpp_n}$} = [&12,4, 24.8, 37.2, 49.6, 62.0, 74.4, 86.8, 99.2, \\&111.6] \pm 18\% \label{e2}
\end{aligned}
\end{equation}
\begin{equation}
\begin{aligned}
\textit{$Q_{M_n}$} = [&\{8,8,8\}, \{8,8,8\}, \{4,4,4\}, \{7,7,7\}, \{4,4,4\}, \\&\{2,2,2\}, \{5,5,5\}, \{4,4,4\}, \{2,2,2\}, \{1,1,1\}] \label{e3}
\end{aligned}
\end{equation}
 
\begin{figure}[htbp]
\centerline{\includegraphics{fig2.png}}
\caption{Flowchart of reconfiguration algorithm.}
\label{fig4}
\end{figure}

Due to 5\% error of MPP current candidates and 18\% error of MPP voltage candidates, total error rate of MPP candidates is 23\%. Though procedure of algorithm, MPP candidates with 23\% are: [\{ 0.5, 0.5, 2\}, \{0.5, 2, 2\}, \{2, 2, 2\}, \{2, 2, 3\}] and require working modules per strings are: [\{8, 8, 8\}, \{7, 7, 7\}, \{5, 5, 5\}, \{4, 4, 4\}]. And PV array are organized as Fig \ref{fig4}
Those MPP candidates are able to generate maximum output power of PV system, but the configurations of MPP candidates may meet electrical connection overhead. In order to reduce electrical connection simulation time, replacement and feasibility check will be approved. 

Consider fourth MPP candidate, it require panels on string one working at 2A, panels on string two working at 2A, panels on string three working at 3A and require 4 PV modules working for each string. Firstly, selecting panels for high current string, string three. According to Table \ref{tab}, only Panel 2, Panel 7 and Panel 9 are able to work at 3A. Those four panel can provide 4 PV module working at 3A and meet requirement of \textit{$Q_{M_n}$}. Because on extra panel can provide module  working on 3A, no replacement will be approved. Means no further improvement for configuration of string three.
Then select panel for string two, Panel 4 and Panel 8 will be selected, but this configuration is not optimized. Selected panels provide 5 modules able to work at 2A, satisfy requirement of \textit{$Q_{M_n}$} $\ge$ 4, but one module more than requirement. The process will replace Panel 4 by Panel 1 that provide just 4 modules working at 2A. Configuration for string two are Panel 8 and Panel 1. At last, select panels for string one. Panel 4, Panel 5, Panel 6 are selected, they can provide 5 modules working at 2A that satisfy requirement. For configuration on string one due to it is the last string to process and meet requirement, in order to reduce computing time replacement will not be approved. Configuration of MPP candidate \{2, 2, 3\} are \textit{St} 1:\{Panel 4,Panel 5, Panel 6\}, \textit{St} 2: \{Panel 1, Panel 8\}, \textit{St} 3: \{Panel 2, Panel 7, Panel 9\}. As described in section \ref{assum}, each string have identical length and Panel 3 not been used in any configuration. So optimized configuration for this MPP candidate are: \textit{St} 1:\{Panel 4,Panel 5, Panel 6\}, \textit{St} 2: \{Panel 1, Panel 8, Panel 3\}, \textit{St} 3: \{Panel 2, Panel 7, Panel 9\}. The MPP values of this configuration calculated by multiplay string voltage and current equal to 347.2W are put into evidence. 

For the third MPP candidate, as same calculating procedure as fourth MPP candidate, optimized configuration for MPP candidate \{2, 2, 2\} are: \textit{St} 1:\{Panel 4,Panel 8, Panel 3\}, \textit{St} 2: \{Panel 1, Panel 2, Panel 6\}, \textit{St} 3: \{Panel 5, Panel 7, Panel 9\}. 

The first MPP candidate \{0.5, 0.5, 2\}, after replacement string three contain four panels, \{Panel 4, Panel 8, Panel 1, Panel 2\} but it is over maximum string length of this PV array.
Also for MPP candidate \{0.5, 2, 2\}, there are no configuration can't provide 7 modules working 
on three different string at 0.5A, 2A, 2A, respectively. 

\section{evaluation of optimization algorithm}
%\subsection{Verification of the Proposed algorithm}
This section verifies the effectiveness of proposed algorithm using LTspice simulation. With pilot example presented in last section, first shading scenario as showed in Fig.\ref{fig5a}. To minimize power loss, the proposed algorithm, as described, is applied to reconfigure the system. The resulting configuration is depicted in Fig.\ref{fig5b}. 

\begin{figure}[htbp]
\begin{minipage}[t]{0.5\linewidth}
\centerline{\includegraphics{fig2.png}}
\caption{(a)}
\label{fig5a}
%\end{minipage}%
%\begin{minipage}[t]{0.5\linewidth}
\centerline{\includegraphics{fig2.png}}
\caption{(b)}
\label{fig5b}
\end{minipage}
\end{figure}

As shown, working module on each string are large or equal than required \textit{$Q_{M_n}$}, which means mismatched power loss are minimized. The \textit{P-V} curve for this PV array before and after reconfiguration are showed in Fig.\ref{fig6}. It can be seen that after reconfiguration maximum power (!!!!!!!!!! W) are significantly higher than before reconfiguration maximum power (!!!!!!!!W).

\begin{figure}[htbp]
\centerline{\includegraphics{fig2.png}}
\caption{Flowchart of reconfiguration algorithm.}
\label{fig6}
\end{figure}
The simulations were performed using an Intel i7-4770 3.2-Ghz processor and 32.0 GB of RAM memory. The calculation time used by the proposed estimation method was 165.2 ms, which is significantly lower than the 57 h, 57 m, and 55 s needed by the exhaustive search algorithm.
\section{conclusions}

%\begin{figure}
%\centering
%\subfigure[the first subfigure]{
%\begin{minipage}[b]{0.2\textwidth}
%\includegraphics[width=1\textwidth]{fig2.png} \\
%\includegraphics[width=1\textwidth]{fig2.png}
%\end{minipage}
%}
%\subfigure[the second subfigure]{
%\begin{minipage}[b]{0.2\textwidth}fig2
%\includegraphics[width=1\textwidth]{fig2.png} \\
%\includegraphics[width=1\textwidth]{fig2.png}
%\end{minipage}
%}
%\end{figure}






%\vspace{100pt}
%Before you begin to format your paper, first write and save the content as a 
%separate text file. Complete all content and organizational editing before 
%formatting. Please note sections \ref{AA}--\ref{SCM} below for more information on 
%proofreading, spelling and grammar.
%
%Keep your text and graphic files separate until after the text has been 
%formatted and styled. Do not number text heads---{\LaTeX} will do that 
%for you.
%
%\subsection{Abbreviations and Acronyms}\label{AA}
%Define abbreviations and acronyms the first time they are used in the text, 
%even after they have been defined in the abstract. Abbreviations such as 
%IEEE, SI, MKS, CGS, ac, dc, and rms do not have to be defined. Do not use 
%abbreviations in the title or heads unless they are unavoidable.
%
%\subsection{Units}
%\begin{itemize}
%\item Use either SI (MKS) or CGS as primary units. (SI units are encouraged.) English units may be used as secondary units (in parentheses). An exception would be the use of English units as identifiers in trade, such as ``3.5-inch disk drive''.
%\item Avoid combining SI and CGS units, such as current in amperes and magnetic field in oersteds. This often leads to confusion because equations do not balance dimensionally. If you must use mixed units, clearly state the units for each quantity that you use in an equation.
%\item Do not mix complete spellings and abbreviations of units: ``Wb/m\textsuperscript{2}'' or ``webers per square meter'', not ``webers/m\textsuperscript{2}''. Spell out units when they appear in text: ``. . . a few henries'', not ``. . . a few H''.
%\item Use a zero before decimal points: ``0.25'', not ``.25''. Use ``cm\textsuperscript{3}'', not ``cc''.)
%\end{itemize}
%
%\subsection{Equations}
%Number equations consecutively. To make your 
%equations more compact, you may use the solidus (~/~), the exp function, or 
%appropriate exponents. Italicize Roman symbols for quantities and variables, 
%but not Greek symbols. Use a long dash rather than a hyphen for a minus 
%sign. Punctuate equations with commas or periods when they are part of a 
%sentence, as in:
%\begin{equation}
%a+b=\gamma\label{eq}
%\end{equation}
%
%Be sure that the 
%symbols in your equation have been defined before or immediately following 
%the equation. Use ``\eqref{eq}'', not ``Eq.~\eqref{eq}'' or ``equation \eqref{eq}'', except at 
%the beginning of a sentence: ``Equation \eqref{eq} is . . .''
%
%\subsection{\LaTeX-Specific Advice}
%
%Please use ``soft'' (e.g., \verb|\eqref{Eq}|) cross references instead
%of ``hard'' references (e.g., \verb|(1)|). That will make it possible
%to combine sections, add equations, or change the order of figures or
%citations without having to go through the file line by line.
%
%Please don't use the \verb|{eqnarray}| equation environment. Use
%\verb|{align}| or \verb|{IEEEeqnarray}| instead. The \verb|{eqnarray}|
%environment leaves unsightly spaces around relation symbols.
%
%Please note that the \verb|{subequations}| environment in {\LaTeX}
%will increment the main equation counter even when there are no
%equation numbers displayed. If you forget that, you might write an
%article in which the equation numbers skip from (17) to (20), causing
%the copy editors to wonder if you've discovered a new method of
%counting.
%
%{\BibTeX} does not work by magic. It doesn't get the bibliographic
%data from thin air but from .bib files. If you use {\BibTeX} to produce a
%bibliography you must send the .bib files. 
%
%{\LaTeX} can't read your mind. If you assign the same label to a
%subsubsection and a table, you might find that Table I has been cross
%referenced as Table IV-B3. 
%
%{\LaTeX} does not have precognitive abilities. If you put a
%\verb|\label| command before the command that updates the counter it's
%supposed to be using, the label will pick up the last counter to be
%cross referenced instead. In particular, a \verb|\label| command
%should not go before the caption of a figure or a table.
%
%Do not use \verb|\nonumber| inside the \verb|{array}| environment. It
%will not stop equation numbers inside \verb|{array}| (there won't be
%any anyway) and it might stop a wanted equation number in the
%surrounding equation.
%
%\subsection{Some Common Mistakes}\label{SCM}
%\begin{itemize}
%\item The word ``data'' is plural, not singular.
%\item The subscript for the permeability of vacuum $\mu_{0}$, and other common scientific constants, is zero with subscript formatting, not a lowercase letter ``o''.
%\item In American English, commas, semicolons, periods, question and exclamation marks are located within quotation marks only when a complete thought or name is cited, such as a title or full quotation. When quotation marks are used, instead of a bold or italic typeface, to highlight a word or phrase, punctuation should appear outside of the quotation marks. A parenthetical phrase or statement at the end of a sentence is punctuated outside of the closing parenthesis (like this). (A parenthetical sentence is punctuated within the parentheses.)
%\item A graph within a graph is an ``inset'', not an ``insert''. The word alternatively is preferred to the word ``alternately'' (unless you really mean something that alternates).
%\item Do not use the word ``essentially'' to mean ``approximately'' or ``effectively''.
%\item In your paper title, if the words ``that uses'' can accurately replace the word ``using'', capitalize the ``u''; if not, keep using lower-cased.
%\item Be aware of the different meanings of the homophones ``affect'' and ``effect'', ``complement'' and ``compliment'', ``discreet'' and ``discrete'', ``principal'' and ``principle''.
%\item Do not confuse ``imply'' and ``infer''.
%\item The prefix ``non'' is not a word; it should be joined to the word it modifies, usually without a hyphen.
%\item There is no period after the ``et'' in the Latin abbreviation ``et al.''.
%\item The abbreviation ``i.e.'' means ``that is'', and the abbreviation ``e.g.'' means ``for example''.
%\end{itemize}
%An excellent style manual for science writers is \cite{b7}.
%
%\subsection{Authors and Affiliations}
%\textbf{The class file is designed for, but not limited to, six authors.} A 
%minimum of one author is required for all conference articles. Author names 
%should be listed starting from left to right and then moving down to the 
%next line. This is the author sequence that will be used in future citations 
%and by indexing services. Names should not be listed in columns nor group by 
%affiliation. Please keep your affiliations as succinct as possible (for 
%example, do not differentiate among departments of the same organization).
%
%\subsection{Identify the Headings}
%Headings, or heads, are organizational devices that guide the reader through 
%your paper. There are two types: component heads and text heads.
%
%Component heads identify the different components of your paper and are not 
%topically subordinate to each other. Examples include Acknowledgments and 
%References and, for these, the correct style to use is ``Heading 5''. Use 
%``figure caption'' for your Figure captions, and ``table head'' for your 
%table title. Run-in heads, such as ``Abstract'', will require you to apply a 
%style (in this case, italic) in addition to the style provided by the drop 
%down menu to differentiate the head from the text.
%
%Text heads organize the topics on a relational, hierarchical basis. For 
%example, the paper title is the primary text head because all subsequent 
%material relates and elaborates on this one topic. If there are two or more 
%sub-topics, the next level head (uppercase Roman numerals) should be used 
%and, conversely, if there are not at least two sub-topics, then no subheads 
%should be introduced.
%
%\subsection{Figures and Tables}
%\paragraph{Positioning Figures and Tables} Place figures and tables at the top and 
%bottom of columns. Avoid placing them in the middle of columns. Large 
%figures and tables may span across both columns. Figure captions should be 
%below the figures; table heads should appear above the tables. Insert 
%figures and tables after they are cited in the text. Use the abbreviation 
%``Fig.~\ref{fig}'', even at the beginning of a sentence.
%
%\begin{table}[htbp]
%\caption{Table Type Styles}
%\begin{center}
%\begin{tabular}{|c|c|c|c|}
%\hline
%\textbf{Table}&\multicolumn{3}{|c|}{\textbf{Table Column Head}} \\
%\cline{2-4} 
%\textbf{Head} & \textbf{\textit{Table column subhead}}& \textbf{\textit{Subhead}}& \textbf{\textit{Subhead}} \\
%\hline
%copy& More table copy$^{\mathrm{a}}$& &  \\
%\hline
%\multicolumn{4}{l}{$^{\mathrm{a}}$Sample of a Table footnote.}
%\end{tabular}
%\label{tab1}
%\end{center}
%\end{table}
%
%\begin{figure}[htbp]
%\centerline{\includegraphics{fig2.png}}
%\caption{Example of a figure caption.}
%\label{fig}
%\end{figure}
%
%Figure Labels: Use 8 point Times New Roman for Figure labels. Use words 
%rather than symbols or abbreviations when writing Figure axis labels to 
%avoid confusing the reader. As an example, write the quantity 
%``Magnetization'', or ``Magnetization, M'', not just ``M''. If including 
%units in the label, present them within parentheses. Do not label axes only 
%with units. In the example, write ``Magnetization (A/m)'' or ``Magnetization 
%\{A[m(1)]\}'', not just ``A/m''. Do not label axes with a ratio of 
%quantities and units. For example, write ``Temperature (K)'', not 
%``Temperature/K''.
%
%\section*{Acknowledgment}
%
%The preferred spelling of the word ``acknowledgment'' in America is without 
%an ``e'' after the ``g''. Avoid the stilted expression ``one of us (R. B. 
%G.) thanks $\ldots$''. Instead, try ``R. B. G. thanks$\ldots$''. Put sponsor 
%acknowledgments in the unnumbered footnote on the first page.
%
%\section*{References}
%
%Please number citations consecutively within brackets . The 
%sentence punctuation follows the bracket. Refer simply to the reference 
%number, as in \cite{b3}---do not use ``Ref. \cite{b3}'' or ``reference \cite{b3}'' except at 
%the beginning of a sentence: ``Reference \cite{b3} was the first $\ldots$''
%
%Number footnotes separately in superscripts. Place the actual footnote at 
%the bottom of the column in which it was cited. Do not put footnotes in the 
%abstract or reference list. Use letters for table footnotes.
%
%Unless there are six authors or more give all authors' names; do not use 
%``et al.''. Papers that have not been published, even if they have been 
%submitted for publication, should be cited as ``unpublished'' \cite{b4}. Papers 
%that have been accepted for publication should be cited as ``in press'' \cite{b5}. 
%Capitalize only the first word in a paper title, except for proper nouns and 
%element symbols.
%
%For papers published in translation journals, please give the English 
%citation first, followed by the original foreign-language citation \cite{b6}.

\begin{thebibliography}{00}
\bibitem{b1} Koutroulis, Eftichios, and Frede Blaabjerg. "A new technique for tracking the global maximum power point of PV arrays operating under partial-shading conditions." IEEE Journal of Photovoltaics 2.2 (2012): 184-190.
\bibitem{b2} La Manna, Damiano, et al. "Reconfigurable electrical interconnection strategies for photovoltaic arrays: A review." Renewable and Sustainable Energy Reviews 33 (2014): 412-426.
\bibitem{b3}Storey, Jonathan, Peter R. Wilson, and Darren Bagnall. "The optimized-string dynamic photovoltaic array." IEEE Transactions on Power Electronics 29.4 (2014): 1768-1776.
\bibitem{b4} Storey, Jonathan P., Peter R. Wilson, and Darren Bagnall. "Improved optimization strategy for irradiance equalization in dynamic photovoltaic arrays." IEEE transactions on power electronics 28.6 (2013): 2946-2956.
\bibitem{b5} P. Carotenuto, A. D. Cioppa, A. Marcelli, and G. Spagnuolo, “An evolu- tionary approach to the dynamical reconfiguration of photovoltaic fields,” Neurocomputing, vol. 170, pp. 393–405, 2015.
\bibitem{b6} Y. Yorozu, M. Hirano, K. Oka, and Y. Tagawa, ``Electron spectroscopy studies on magneto-optical media and plastic substrate interface,'' IEEE Transl. J. Magn. Japan, vol. 2, pp. 740--741, August 1987 [Digests 9th Annual Conf. Magnetics Japan, p. 301, 1982].
\bibitem{b7} Orozco-Gutierrez, M. L., et al. "Fast estimation of MPPs in mismatched PV arrays based on lossless model." Clean Electrical Power (ICCEP), 2015 International Conference on. IEEE, 2015.
\bibitem{b8} Spagnuolo, Giovanni, et al. "Control of photovoltaic arrays: Dynamical reconfiguration for fighting mismatched conditions and meeting load requests." IEEE Industrial Electronics Magazine 9.1 (2015): 62-76.
\bibitem{b9} Carotenuto, Pietro Luigi, et al. "Online recording a PV module fingerprint." IEEE Journal of Photovoltaics 4.2 (2014): 659-668.
\bibitem{b10} Orozco-Gutierrez, M. L., et al. "Optimized configuration of mismatched photovoltaic arrays." IEEE J. Photovolt 6.5 (2016): 1210-1220.
  
\end{thebibliography}
%\vspace{12pt}
%\color{red}
%IEEE conference templates contain guidance text for composing and formatting conference papers. Please ensure that all template text is removed from your conference paper prior to submission to the conference. Failure to remove the template text from your paper may result in your paper not being published.

\end{document}
